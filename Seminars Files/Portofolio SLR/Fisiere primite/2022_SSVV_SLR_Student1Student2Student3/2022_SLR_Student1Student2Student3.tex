%% 
%% Copyright 2019-2021 Elsevier Ltd
%% 
%% This file is part of the 'CAS Bundle'.
%% --------------------------------------
%% 
%% It may be distributed under the conditions of the LaTeX Project Public
%% License, either version 1.2 of this license or (at your option) any
%% later version.  The latest version of this license is in
%%    http://www.latex-project.org/lppl.txt
%% and version 1.2 or later is part of all distributions of LaTeX
%% version 1999/12/01 or later.
%% 
%% The list of all files belonging to the 'CAS Bundle' is
%% given in the file `manifest.txt'.
%% 
%% Template article for cas-dc documentclass for 
%% double column output.




\documentclass[a4paper,fleqn]{cas-dc}

% If the frontmatter runs over more than one page
% use the longmktitle option.

%\documentclass[a4paper,fleqn,longmktitle]{cas-dc}

%\usepackage[numbers]{natbib}
%\usepackage[authoryear]{natbib}
\usepackage[authoryear,longnamesfirst]{natbib}

%%%Author macros
\def\tsc#1{\csdef{#1}{\textsc{\lowercase{#1}}\xspace}}
\tsc{WGM}
\tsc{QE}
%%%

% Uncomment and use as if needed
%\newtheorem{theorem}{Theorem}
%\newtheorem{lemma}[theorem]{Lemma}
%\newdefinition{rmk}{Remark}
%\newproof{pf}{Proof}
%\newproof{pot}{Proof of Theorem \ref{thm}}


\usepackage{color}
\definecolor{Gray}{gray}{0.9}
\definecolor{LightCyan}{rgb}{0.88,1,1}
\definecolor{darkmagenta}{rgb}{0.55, 0.0, 0.55}
\definecolor{dartmouthgreen}{rgb}{0.05, 0.5, 0.06}
\definecolor{deepblue}{rgb}{0,0,0.5}
\definecolor{deepgreen}{rgb}{0,0.5,0}

\begin{document}
\let\WriteBookmarks\relax
\def\floatpagepagefraction{1}
\def\textpagefraction{.001}

% Short title
\shorttitle{<short title of the paper for running head>}    

% Short author
\shortauthors{<short author list for running head>}  

% Main title of the paper
\title [mode = title]{<main title>}  

% Title footnote mark
% eg: \tnotemark[1]
\tnotemark[<tnote number>] 

% Title footnote 1.
% eg: \tnotetext[1]{Title footnote text}
\tnotetext[<tnote number>]{<tnote text>} 

% First author
%
% Options: Use if required
% eg: \author[1,3]{Author Name}[type=editor,
%       style=chinese,
%       auid=000,
%       bioid=1,
%       prefix=Sir,
%       orcid=0000-0000-0000-0000,
%       facebook=<facebook id>,
%       twitter=<twitter id>,
%       linkedin=<linkedin id>,
%       gplus=<gplus id>]

\author[<aff no>]{<author name>}%[<options>]

% Corresponding author indication
\cormark[<corr mark no>]

% Footnote of the first author
\fnmark[<footnote mark no>]

% Email id of the first author
\ead{<email address>}

% URL of the first author
\ead[url]{<URL>}

% Credit authorship
% eg: \credit{Conceptualization of this study, Methodology, Software}
\credit{<Credit authorship details>}

% Address/affiliation
\affiliation[<aff no>]{organization={},
            addressline={}, 
            city={},
%          citysep={}, % Uncomment if no comma needed between city and postcode
            postcode={}, 
            state={},
            country={}}

\author[<aff no>]{<author name>}%[<options>]

% Footnote of the second author
\fnmark[2]

% Email id of the second author
\ead{}

% URL of the second author
\ead[url]{}

% Credit authorship
\credit{}

% Address/affiliation
\affiliation[<aff no>]{organization={},
            addressline={}, 
            city={},
%          citysep={}, % Uncomment if no comma needed between city and postcode
            postcode={}, 
            state={},
            country={}}

% Corresponding author text
\cortext[1]{Corresponding author}

% Footnote text
\fntext[1]{}

% For a title note without a number/mark
%\nonumnote{}

% Here goes the abstract
\begin{abstract}
ABSTRACT 
\end{abstract}

% Use if graphical abstract is present
%\begin{graphicalabstract}
%\includegraphics{}
%\end{graphicalabstract}

% Research highlights
%\begin{highlights}
%\item 
%\item 
%\item 
%\end{highlights}

% Keywords
% Each keyword is seperated by \sep
\begin{keywords}
 \sep \sep \sep
\end{keywords}

\maketitle

% Main text


\section{Systematic literature review (SLR)}


\textcolor{darkmagenta}{Please read the references \cite{Kitchenham2007} and \cite{Kitchenham2004}.}

\textcolor{darkmagenta}{Chapter 3 from this reference \cite{Campeanu2018} provides you a concrete example of how a SLR was conducted.}

\textcolor{darkmagenta}{You can also study this article that contains also a SLR \cite{DavilaSLR2021}.}

\section{Study design}

\subsection{Review need identification}

\subsection{Research questions definition}

\subsection{Protocol definition}


\section{Conducting the SLR}
\subsection{Search and selection process}
\subsubsection{Database search}
\subsubsection{Merging, and duplicates and impurity removal}
\subsubsection{Application of the selection criteria}    

\subsection{Data extraction}

\textcolor{darkmagenta}{Here you detail in a paragraph how many items you have selected and then add them to the table  \ref{TabelArticoleSelectate}}

\begin{table*}[htbp]
    \begin{center}
    \begin{tabular}{ |c|c|c|c| } 
    \hline
    \textbf{Id} & \textbf{Citarea} & \textbf{Titlul, Autori} & \textbf{An} \\
    \hline
    P1 & cu cite & titlul & anul \\
    \hline
    P2 & cu cite & titlul & anul \\
    \hline
    P10 & cu cite & titlul & anul \\
    \hline
    \end{tabular}
    \end{center}
   \caption{Lista articolelor selectate }
    \label{TabelArticoleSelectate}
\end{table*}

\subsection{Data synthesis}
\label{SectiuneSintezaArticole}

\textcolor{darkmagenta}{Here, for each of the 10 selected articles, a short summary of 2-3 paragraphs is made.} 


\textcolor{darkmagenta}{Add that table to include the 10 articles and different features you are looking for in the existing approaches so that the table shows which are the most used approaches, different methods used, benckmarks / dataset / case studies used, etc ... } 


\textcolor{darkmagenta}{To add a summary table with certain criteria for each selected article ...} 

\textcolor{magenta}{A table model \ref{TabelSumativ}, but you adapt it according to the criteria established on the researched field ....}

\begin{table*}[h]
\begin{center}
    \begin{tabular}{ |c|c|c|c| } 
    \hline
    \textbf{Articol} & \textbf{Abordare} & \textbf{Tools} & \textbf{Case study}\\
    \hline
    Articol 1 & abordare 1  & tool 1  & academic\\
      \hline
    Articol 2 & abordare 2 & tool 1  & academic\\
      \hline
    Articol 3 & abordare 1 & tool 1  & academic\\
      \hline
    Articol .... & abordare 1 & tool 1  & academic\\
      \hline
    Articol 10 & abordare 3 & tool 1  & real life\\
    \hline
    \end{tabular}
    \end{center}
    \caption{ Tabel sumativ cu abordari }
    \label{TabelSumativ}
\end{table*}    

\section{Results}

\textcolor{darkmagenta}{The research questions are repeated and an answer is offered.}




% Numbered list
% Use the style of numbering in square brackets.
% If nothing is used, default style will be taken.
%\begin{enumerate}[a)]
%\item 
%\item 
%\item 
%\end{enumerate}  

% Unnumbered list
%\begin{itemize}
%\item 
%\item 
%\item 
%\end{itemize}  

% Description list
%\begin{description}
%\item[]
%\item[] 
%\item[] 
%\end{description}  

% Figure
%\begin{figure}[<options>]
%	\centering
%		\includegraphics[<options>]{}
%	  \caption{}\label{fig1}
%\end{figure}


%\begin{table}[<options>]
%\caption{}\label{tbl1}
%\begin{tabular*}{\tblwidth}{@{}LL@{}}
%\toprule
%  &  \\ % Table header row
%\midrule
% & \\
% & \\
% & \\
% & \\
%\bottomrule
%\end{tabular*}
%\end{table}

% Uncomment and use as the case may be
%\begin{theorem} 
%\end{theorem}

% Uncomment and use as the case may be
%\begin{lemma} 
%\end{lemma}

%% The Appendices part is started with the command \appendix;
%% appendix sections are then done as normal sections
%% \appendix

%\section{}\label{}

% To print the credit authorship contribution details
\printcredits

%% Loading bibliography style file
%\bibliographystyle{model1-num-names}
%\bibliographystyle{cas-model2-names}
\bibliographystyle{2022_References}

% Loading bibliography database
\bibliography{}

% Biography
\bio{}
% Here goes the biography details.
\endbio

%\bio{pic1}
% Here goes the biography details.
\endbio

\end{document}

